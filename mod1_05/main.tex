\documentclass{article}
\usepackage[utf8]{inputenc}
\usepackage{multirow}
\usepackage{todonotes}
\newtheorem{domanda}{Domanda}[section]

\title{Tutorato Architettura degli Elaboratori Modulo 1 \\ Lezione 5}
\author{Francesco Pelosin}
\date{20 December 2019}

\begin{document}

\maketitle

\section{Calcolo delle Prestazioni}

\begin{itemize}

     \item $F$: frequenza del ciclo di clock. Equivale al numero di volte che il ciclo di clock si ripete in un secondo $F=\frac{1}{T}$.

     \item $T$: periodo del ciclo di clock. Equivale al tempo di durata di un ciclo di clock, è il reciproco della frequenza di clock: $T = \frac{1}{F}$.

     \item $CPI_{m}$: (Cycles Per Instruction) numero di cicli di clock impiegati per eseguire una particolare istruzione su una data macchina $m$.

     \item $CPI_{m,c}$: numero di cicli medi data una certa macchina $m$ ed un certo compilatore $c$. Equivale alla media pesata del $CPI_{m}$ di ciascuna istruzione per la distribuzione delle istruzioni del compilatore.

     \item $IC$: (Instruction Count) numero di istruzioni di un programma.

     \item $ExeTime_{m}$: (Execution Time) tempo di esecuzione di un particolare programma su una particolare macchina $m$. Equivale al numero di cicli medi per istruzione $CPI_{m,c}$ per il numero di istruzioni del programma $IC$, per il periodo del ciclo.

     $$ ExeTime_{m} = IC \cdot  CPI_{m,c} \cdot  T = IC \cdot  CPI_{m,c} \cdot  \frac{1}{F}$$

     \item $Perf_{m}$: performance di una particolare macchina $m$. Una prestazione migliore corrisponde ad un tempo di esecuzione più breve, viene adottato quindi, come misura della performance di una macchina, l'inverso del tempo di esecuzione.

     $$Perf_{m} = \frac{1}{ExeTime_{m}}$$

     \newpage
     
     \item $Speedup(m_1,m_2)$: Misura quanto un sistema $m_1$ è più veloce di un sistema $m_2$. 
     
     $$Speedup(m_1,m_2) = \frac{Perf_{m_1}}{Perf_{m_2}} = \frac{ExeTime_{m_2}}{ExeTime_{m_1}}$$
    
    \item MIPS: milioni di istruzioni per secondo, $MIPS = \frac{IC}{ExeTime \cdot 10^{6}} = \dots = \frac{F}{CPI \cdot 10^6}$. 

\end{itemize}



\subsection{Esercizo Acquisto Macchina}
Si considerino due diverse macchine: $M_1$ ed $M_2$, aventi lo stesso set di istruzioni, partizionato in tre classi $A,B,C$.
    
    
\begin{table}[h!]
\begin{tabular}{c|c|c|c|c|c|}
\cline{2-6}
                          & $CPI_{M_1}$ & $CPI_{M_2}$ & Utilizzo $C_1$ & Utilizzo $C_2$ & Utilizzo $C_3$ \\ \hline
\multicolumn{1}{|c|}{$A$} & 4           & 2           & 30\%           & 30\%           & 50\%           \\ \hline
\multicolumn{1}{|c|}{$B$} & 6           & 4           & 50\%           & 20\%           & 30\%           \\ \hline
\multicolumn{1}{|c|}{$C$} & 8           & 3           & 20\%           & 50\%           & 20\%           \\ \hline
\end{tabular}
\end{table}
    
    \begin{itemize}
        \item $M_1$ ha una frequenza di clock pari a $400$ MHz
        \item $M_2$ ha una frequenza di clock pari a $200$ MHz
        \item $C_1$ è un compilatore sviluppato dai produttori di $M_1$
        \item $C_2$ è un compilatore sviluppato dai produttori di $M_2$
        \item $C_3$ è un compilatore sviluppato da un terzo costruttore.
    \end{itemize}
    
    Si supponga che il codice prodotto dai tre compilatori per uno stesso programma preveda un numero uguale  di istruzioni eseguite ($IC$), ma una diversa distribuzione nelle varie classi come riportato nella tabella.
    
    \begin{domanda}
    Usando $C_1$ su $M_1$ e $M_2$, di quanto $M_1$ è più veloce di $M_2$?
    \end{domanda}
    \paragraph{Soluzione}
    L'esercizio ci chiede di calcolare lo $Speedup(M_1, M_2)$. Iniziamo, dunque, calcolando il tempo di esecuzione di $M_1$ e $M_2$ usando il compilatore $C_1$. Per fare questo ci serve il $CPI_{M_1, C_1}$ e $CPI_{M_2, C_1}$ nei due casi.
    
    $$CPI_{M_1, C_1} = 4 \cdot  0.3 + 6\cdot 0.5 + 8\cdot 0.2 = 1.2 + 3 + 1.6 = 5.8$$
    $$CPI_{M_2, C_1} = 2 \cdot  0.3 + 4\cdot 0.5 + 3\cdot 0.2 = 0.6 + 2 + 0.6 = 3.2$$
    
    Il tempo di esecuzione di $M_1$ è:
    
    $$ExeTime_{M_1} = IC\cdot CPI_{M_1,C_1} \cdot  \frac{1}{F_{M_{1}}} = \frac{IC\cdot 5.8}{400\cdot 10^{6}}$$
    ossia:
    $$Perf_{M_{1}} = \frac{400\cdot 10^{6}}{IC\cdot 5.8} $$
    
    Il tempo di esecuzione di $M_2$ è:
    
    $$ExeTime_{M_2} = IC\cdot CPI_{M_2,C_1} \cdot  \frac{1}{F_{M_{2}}} = \frac{IC\cdot 3.2}{200\cdot 10^{6}}$$
    ossia:
    $$Perf_{M_{2}} = \frac{200\cdot 10^{6}}{IC\cdot 3.2} $$
    
    Ora abbiamo tutti i dati necessari per rispondere alla domanda:
    
    $$Speedup(M_1, M_2) = \frac{Perf_{M_1}}{Perf_{M_2}} = \frac{400\cdot 10^6}{IC \cdot 5.8} \cdot  \frac{IC \cdot 3.2}{200\cdot 10^6}$$
    
    semplificando avremo:
    
    $$Speedup(M_1, M_2) = \frac{6.4}{5.8} = 1.10$$
    
    Utilizzando il compilatore $C_1$ il sistema $M_1$ è $1.10$ volte più performante di $M_2$.
    
    \begin{domanda}
    Usando $C_2$ su $M1$ e $M_2$, quanto più veloce è $M_2$ rispetto a $M_1$? 
    \end{domanda}
    
    \paragraph{Soluzione}
    Usiamo la distribuzione delle istruzioni del compilatore $C_2$ e procediamo in maniera analoga all'esercizio precedente:
    
    $$CPI_{M_1, C_2} = 4 \cdot  0.3 + 6\cdot 0.2 + 8\cdot 0.5 = 1.2 + 1.2 + 4 = 6.4$$
    $$CPI_{M_2, C_2} = 2 \cdot  0.3 + 4\cdot 0.2 + 3\cdot 0.5 = 0.6 + 0.8 + 1.5 = 2.9$$
    
    Il tempo di esecuzione di $M_1$ è:
    
    $$ExeTime_{M_1} = IC\cdot CPI_{M_1,C_2} \cdot  \frac{1}{F_{M_{1}}} = \frac{IC\cdot 6.4}{400\cdot 10^{6}}$$
    ossia:
    $$Perf_{M_{1}} = \frac{400\cdot 10^{6}}{IC\cdot 6.4} $$
    
    Il tempo di esecuzione di $M_2$ è:
    
    $$ExeTime_{M_2} = IC\cdot CPI_{M_2,C_2} \cdot  \frac{1}{F_{M_{2}}} = \frac{IC\cdot 2.9}{200\cdot 10^{6}}$$
    ossia:
    $$Perf_{M_{2}} = \frac{200\cdot 10^{6}}{IC\cdot 2.9} $$
    
    Ora abbiamo tutti i dati necessari per rispondere alla domanda:
    
    $$Speedup(M_2, M_1) = \frac{Perf_{M_2}}{Perf_{M_1}} = \frac{200\cdot 10^6}{IC\cdot 2.9} \cdot  \frac{IC\cdot 6.4}{400\cdot 10^6}$$
    
    semplificando avremo:
    
    $$Speedup(M_2, M_1) = \frac{6.4}{5.8} = 1.10$$
    
    Utilizzando il compilatore $C_2$ il sistema $M_2$ risulta $1.10$ volte più performante dim $M_1$.   

    \begin{domanda}
    Se si acquista $M_1$, quale dei tre compilatori conviene usare?
    Se si acquista $M_2$, quale dei tre compilatori conviene usare?
    \end{domanda}
    
    Calcoliamo le prestazionei del compilatore $C_3$:
    
    $$CPI_{M_1,C_3} = 4\cdot 0.5 + 6\cdot 0.3 + 8\cdot 0.2 = 2+1.8+1.6 = 5.4$$
    $$CPI_{M_2,C_3} = 2\cdot 0.5 + 4\cdot 0.3 + 3\cdot 0.2 = 1+1.2+0.6 = 2.8$$
    
    Tabella delle $CPI$:


\begin{table}[h!]
\begin{tabular}{l|c|c|c|}
\cline{2-4}
                            & $C_1$ & $C_2$ & $C_3$ \\ \hline
\multicolumn{1}{|l|}{$M_1$} & 5.8   & 6.4   & 5.4   \\ \hline
\multicolumn{1}{|l|}{$M_2$} & 3.2   & 2.9   & 2.8   \\ \hline
\end{tabular}
\end{table}

    Dalla tabella si può vedere che in tutti i casi il compilatore $C_3$ ha una $CPI$ media più bassa degli altri, quindi è da preferire sia per $M_1$ che per $M_2$.
    
    \begin{domanda}
    Considerando che $C_3$ si è rivelato il miglior compilatore. Quale macchina costituirà il miglior acquisto, supponendo che tutti gli altri criteri siano identici, compreso il prezzo?
    \end{domanda}
    
    \paragraph{Risposta}
    Bisogna confrontare le prestazioni delle due macchine utilizzando, ovviamente, il compilatore $C_3$. Risulta:
    
    $$ExeTime_{M_1} = IC\cdot CPI_{M_1,C_3} \cdot  \frac{1}{F_{M_{1}}} = \frac{IC\cdot 5.4}{400\cdot 10^{6}}$$
    ossia:
    $$Perf_{M_{1}} = \frac{400\cdot 10^{6}}{IC\cdot 5.4} $$
    
    Il tempo di esecuzione di $M_2$ è:
    
    $$ExeTime_{M_2} = IC\cdot CPI_{M_2,C_3} \cdot  \frac{1}{F_{M_{2}}} = \frac{IC\cdot 2.8}{200\cdot 10^{6}}$$
    ossia:
    $$Perf_{M_{2}} = \frac{200\cdot 10^{6}}{IC\cdot 2.8} $$
    
    Compariamo le macchine:
    
    $$Speedup(M_1, M_2) = \frac{Perf_{M_1}}{Perf_{M_2}} = \frac{400\cdot 10^6}{IC\cdot 5.4} \cdot  \frac{IC\cdot 2.8}{200\cdot 10^6}$$
    
    semplificando avremo:
    
    $$Speedup(M_1, M_2) = \frac{5.6}{5.4} = 1.04$$
    
    Utilizzando il compilatore $C_3$ il sistema $M_1$ risulta più performante 1.04 volte rispetto la macchina $M_2$ e quindi costituisce iil miglior acquisto.
    
    \subsection{Esercizio: Confronto tra Macchine}
    Considerare due macchine $M_1$ (a $300$ MHz) ed $M_2$ (a $450$ MHz) con le seguenti caratteristiche (rispetto a dei compilatori prefissati).
    
    
\begin{table}[h!]
\begin{tabular}{|c|c|c|c|}
\hline
Macchina               & Classe & $CPI$ & Distribuzione \\ \hline
\multirow{4}{*}{$M_1$} & $A$      & 1     & 40\%          \\ \cline{2-4} 
                       & $B$      & 2     & 30\%          \\ \cline{2-4} 
                       & $C$      & 3     & 20\%          \\ \cline{2-4} 
                       & $D$      & 4     & 10\%          \\ \hline
\multirow{2}{*}{$M_2$} & $A$      & 1     & 40\%          \\ \cline{2-4} 
                       & $B$      & 2     & 60\%          \\ \hline
\end{tabular}
\end{table}
    
    Considerare che per uno stesso programma, in media il codice prodotto per $M_2$ prevede il doppio di istruzioni rispetto a quello prodotto per $M_1$.
    
    \begin{domanda}
    Stabilire la macchina migliore.
    \end{domanda}
    
    \paragraph{Soluzione}
    Il numero di istruzioni della macchina $M_2$ per un programma qualsiasi è mediamente il doppio delle istruzioni della macchina $M_1$, possiamo affermare:
    
    $$IC_{M_2} = 2 \cdot IC_{M_1} $$

    Calcoliamo il tempo di esecuzione delle due macchine e poi confrontiamo le performances:
    
    $$CPI_{M_1}  = 1 \cdot 0.4 + 2 \cdot 0.3 + 3 \cdot 0.2 + 4 \cdot 0.1 = 2$$
    
    $$CPI_{M_2}  = 1 \cdot 0.4 + 2 \cdot 0.6 = 1.6$$
    
    $$Perf_{M_1} = \frac{300 \cdot 10^6}{IC_{M_1} \cdot 2}$$

    $$Perf_{M_2} = \frac{450 \cdot 10^6}{2\cdot IC_{M_1} \cdot 1.6}$$
    
    $$Speedup(M_1, M_2) = \frac{Perf_{M_1}}{Perf_{M_2}} = \frac{300 \cdot 10^6}{IC_{M_1} \cdot 2} \cdot \frac{2 \cdot IC_{M_1} \cdot 1.6}{450 \cdot 10^6} $$
    
    semplificando segue:
    
    $$Speedup(M_1, M_2) = \frac{1.6}{1.5} = 1.067$$
    
    Ne segue che la macchina $M_1$ è $1.067$ volte più performante della macchina $M_2$.
    
    %\begin{domanda}
    %Se si usa un nuovo compilatore su $M_2$ che riduce di $\frac{1}{3}$ il numero delle operazioni di tipo $B$, cosa succede alle prestazioni?
    %\end{domanda}
    
    %\paragraph{Soluzione}
    %\todo{Non molto chiaro}
    %Un programma che con $M_1$ ha $IC_{M_1}$ istruzioni con $M_2$ avrà:
    
    %$$IC_{M_2} = 2 \cdot IC_{M_1} \cdot 0.4 + 2 \cdot IC_{M_1} \cdot 0.6 \cdot \frac{2}{3} = 1.6 \cdot IC_{M_1}$$
    
    %La distribuzione si calcola come segue: dato un programma che col precedente compilatore aveva $IC_{M}$ istruzioni, col nuovo compilatore avrà:
    
    %$$IC_{M_2} = 0.4 \cdot IC_{M} + 0.6 \cdot \frac{2}{3} \cdot IC_{M} = 0.8 \cdot IC_{M}$$
    
    %La distribuzione quindi sarà:
    
    %\begin{table}[h!]
    %\begin{tabular}{|c|c|c|}
    %\hline
    %\textbf{Classe} & \textbf{Cicli} & \textbf{Distribuzione}                               \\ \hline
    %$A$             & 1              & $\frac{0.4 * IC_{M}}{0.8 * IC_{M}} \cdot %100 = 50\%$ \\ \hline
    %$B$             & 2              & $\frac{0.4 * IC_{M}}{0.8 * IC_{M}} \cdot %100 = 50\%$ \\ \hline
    %\end{tabular}
    %\end{table}

    %La $CPI_{M_1}$ col nuovo compilatore sarà:
    
    %$$CPI_{M_1} = 1 \cdot 0.5 + 2 \cdot 0.5 = 1.5$$
    
    %confrontiamo le due macchine:
    
    %$$Speedup(M_2, M_1) = \frac{Perf_{M_2}}{Perf_{M_1}} = \frac{450 \cdot 10^6}{1.6 \cdot IC_{M_1} \cdot 1.5} \cdot \frac{2 \cdot IC_{M_1}}{300 \cdot 10^6}$$

    %$$Speedup(M_2, M_1) = \frac{9}{7.2} = 1.25$$
    
    %Concludendo, la macchina $M_2$ in questo caso è migliore della macchina $M_1$.
    
    \subsection{Esercizio: Lunghezza del codice}
    
    Considerare due macchine $M_1$ (a $3$ GHz) ed $M_2$ (a $1.5$ GHz). La seguente tabella illustra le classi di istruzioni macchina, i relativi $CPI$ medi e la distribuzione di probabilità delle istruzioni generate dallo stesso compilatore ($C$).
    
\begin{table}[h!]
\begin{tabular}{|c|c|c|c|}
\hline
Macchina               & Classe & $ICP$ & Distribuzione \\ \hline
\multirow{4}{*}{$M_1$} & $A$    & 1     & 40\%          \\ \cline{2-4} 
                       & $B$    & 2     & 30\%          \\ \cline{2-4} 
                       & $C$    & 3     & 20\%          \\ \cline{2-4} 
                       & $D$    & 4     & 10\%          \\ \hline
\multirow{2}{*}{$M_2$} & $A$    & 1     & 60\%          \\ \cline{2-4} 
                       & $B$    & 2     & 40\%          \\ \hline
\end{tabular}
\end{table}

\begin{domanda}
Supponendo di sapere che il codice prodotto, compilando lo stesso programma per le due
piattaforme, ha identiche prestazioni sia per $M_1$ che per $M_2$, calcolare in che rapporto devono stare i numeri di istruzioni prodotte per i due programmi (ovvero, $IC_{M_1}$ e $IC_{M_2}$) perché ciò si verifichi.
\end{domanda}

\paragraph{Soluzione}

Dobbiamo calcolare il $CPI$ medio per le due macchine.

$$CPI_{M_1,C} = 1 \cdot 0.4 + 2 \cdot 0.3 + 3 \cdot 0.2 + 4 \cdot 0.1 = 2$$

$$CPI_{M_2,C} = 1 \cdot 0.6 + 2 \cdot 0.4 = 1.4$$

I tempi di esecuzione si esprimono come segue:

$$ExeTime_{M_1} = \frac{CPI_{M_1,C} \cdot IC_{M_1}}{F_{M_1}} = \frac{2 \cdot IC_{M_1}}{3 \cdot 10^9} = 0.66 \cdot 10^{-9} \cdot IC_{M_1}$$

$$ExeTime_{M_2} = \frac{CPI_{M_2,C} \cdot IC_{M_2}}{F_{M_2}} = \frac{1.4 \cdot IC_{M_2}}{1.5 \cdot 10^9} = 0.93 \cdot 10^{-9} \cdot IC_{M_2}$$

Affinchè le prestazioni siano le stesse i tempi di esecuzione dovranno essere uguali per cui:

$$ExeTime_{M_1} = ExeTime_{M_2}$$
$$0.66 \cdot 10^{-9} \cdot IC_{M_1} = 0.93 \cdot 10^{-9} \cdot IC_{M_2}$$
$$IC_{M_1} = \frac{0.93}{0.66} \cdot IC_{M_2}$$
$$IC_{M_1} = 1.409 \cdot IC_{M_2}$$

%\subsection{Esercizio: $CPI$ medio ideale (1)}
%Un computer a $1$ GHz, nell’eseguire un certo programma, ha una prestazione ideale di $500$ MIPS.

%\begin{domanda}
%Calcolare il $CPI$ medio ideale.
%\end{domanda}

%\paragraph{Soluzione}
%In un secondo il computer esegue $0.5$ G istruzioni quindi abbiamo: $0.5 \times 10^9 = \frac{10^9}{CPI}$ da cui $CPI = 2$. \todo{Non chiaro}


%\subsection{Esercizio: $CPI$ medio ideale}
%Considerare l'esecuzione di un programma $P$ su una data $CPU$.

%\begin{domanda}
%Calcolare il $CPI$ ideale, considerando che il $CPI$ medio delle load/store è 4.5, il $CPI$ medio delle altre istruzioni è 2, mentre la percentuale di load/store è del 40\%.
%\end{domanda}

%\paragraph{Soluzione}
%Riscriviamo i dati nella seguente tabella:

%\begin{table}[h!]
%\begin{tabular}{|c|c|c|}
%\hline
%Classe     & $CPI$ & Distribuzione \\ \hline
%load/store & 4.5   & 40\%          \\ \hline
%altre      & 2     & 60\%          \\ \hline
%\end{tabular}
%\end{table}

%Di conseguenza:

%$$CPI_{Id} = 0.4 \cdot CPI_{l/s} + 0.6 \cdot CPI_{altre} = 0.4 \cdot 4.5 + 0.6 \cdot 2 = 3$$

%\begin{domanda}
%Calcolare i tempi ideali per eseguire $P$ considerando che $IC = 200 \cdot 10^6$, mentre la %frequenza della $CPU$ è di $500 MHz$.
%\end{domanda}

%\paragraph{Soluzione}
%Il tempo di ciclo (periodo) è:

%$$T= \frac{1}{500} \cdot 10^{-6} = 2 \cdot 10^{-9}$$

%Il tempo ideale di esecuzione del programma $P$ è quindi:

%$$ExeTime = CPI_{id} \cdot IC \cdot T = 3 \cdot 200 \cdot 10^6 \cdot 2 \cdot 10^{-9} = 1.2 sec$$

\end{document}
